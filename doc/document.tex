\documentclass{article}

\usepackage{amsmath}
\usepackage{geometry}
\usepackage{tikz}

\newgeometry{margin=1.25in}

\newcommand{\boundedBy}[1]{\mathcal{O}\left(#1\right)}

\begin{document}

\author{Garrett Lewellen}
\title{A Greedy Polynomial Time Approxmation Scheme for the Linear Assignment Problem}

\maketitle

\section{Introduction}

\paragraph{} Solutions to the the Linear Assignment Problem (LAP) are perfect minimum weighted bipartite matching between two sets. \cite{} explored the problem in terms of workers and tasks whereby each worker is assigned to a unique task such that the overall cost of workers performing tasks is minimized. Beyond the originally investigated application, the LAP appears in coregistration \cite{},  X \cite{}, and Y \cite{}. \cite{} Originally proposed a $\boundedBy{n^4}$ algorithm to be performed by clerks by hand, and later \cite{} developed a $\boundedBy{n^3}$ algorithm meant to be carried out by computers. Further research explores parallel algorithms such as the auction algorithm, \cite{} and algorithms based on statistical mechanics such as the cavity method \cite{}. Here we investigate a greedy $\alpha$-approximation algorithm for the problem whose runtime is $\boundedBy{n^2 \log n}$. Derivation of the algorithm's approximation factor will be given with empircal evaluation against the Hungarian method.

\section{Outline of Algorithm}

Assume an $n \times n$ cost matrix.

\begin{enumerate}
	\item Allocate continuous $n^2$ nodes
	\item In $O(n^2)$ time link the nodes together
	\item Sort the $n^2$ nodes in ascending value in $O(n^2 \log n)$ time
	\item Let cursor = minimum valued node
	\item While cursor is not NULL -- This is $O(n)$
	\begin{enumerate}
		\item Assign cursor job to worker
		\item Store the next next cursor
		\item Enqueue (cursor, nil) into Q
		\item While( Q is not empty )  --- This is $O(2n - 1)$
		\begin{enumerate}
			\item Dequeue (q, a) from Q
			\item Set q's left neighbor to point to q's right neighbor (vice versa)
			\item Set q's top neighbor to point to q's bottom neighbor (vice versa)
			\item If a is nil and for each present left, right, top, or bottom neighbor, enqueue (neighbor, direction)
			\item If a is not nill, then if a is present then enqueue (neighbor, a)
			\item Set q's previous to point to q's next (vice versa)
			\item If cursor equals q, update next cursor to equal q's next
			\item delete q
		\end{enumerate}
		\item cursor = next cursor
	\end{enumerate}
\end{enumerate}

Inner loop analysis showing that you end up doing $O(n^2)$ work -- $n$ on the outer loop and $2n - 1$ on the inner loop.

\begin{equation}
	\sum_{k = 1}^{n} 2k - 1 = 2 \frac{n (n + 1)}{2} - n = n^2
\end{equation}

\paragraph{TODO:}

\begin{enumerate}
	\item Referencing the people who came up with $\pi^2/6$ for $n \to \infty$ on iid exp(1) values
	\item Derive what the expected minimum should be for:
	\begin{enumerate}
		\item At random- Identify probability distribution and/or technique for figuring out the expected sum of picking $n^2$ iid rvs out of $n^2$
		\item Greedy heuristic - Think you'll want to look at the expected minimum out of $n^2$ iid rvs; then take the sum of that over $i = 1 \cdots n$
	\end{enumerate}
	\item After doing the above, then you should be able to make a claim that $\alpha$ ( approx cost / optimal cost ) should trend towards $[1.05, 1.10]$; use plot you put together to verify empirically. 
	\item Decide if it is worthwhile to look at normalized approx cost (approx cost - optimal min cost) / (optimal max cost - optimal min cost). Zero means an algorithm is picking the optimal cost; One means it is antagonistically picking the worst cost (the maximum); and that 1/2 is picking at random.
	\item Discuss how on the greedy heuristic you could end up with different results based on ties. (ie there exists a minimum value, but it appears multiple times in the matrix- which you choose impacts the overall outcome. May lead to another greedy impl with back tracking.)
	\item Runtime comparison of the naive, efficient, and hungarian methods

	\item Decide if this paper is enough to stand on its own, or if it should be part of the GPS trajectory tracking project.

\end{enumerate}

\section{Evaluation}

\section{Conclusion}

\begin{figure}
	\centering
	\caption{Approximation factor}
	\input{obj/approximationFactor.tex}
	\label{fig:approximationFactor}
\end{figure}

\begin{figure}
	\centering
	\caption{Runtime}
	\input{obj/runtime.tex}
	\label{fig:runtime}
\end{figure}

\end{document}
