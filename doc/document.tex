\documentclass{article}

\usepackage{amsmath}
\usepackage{geometry}
\usepackage{tikz}

\newgeometry{margin=1.25in}

\begin{document}

\author{Garrett Lewellen}
\title{A Greedy Polynomial Time Approxmation Scheme for the Linear Assignment Problem}

\maketitle

\section{Introduction}

Simplex method

Hungerian algorithm - Kuhn, Munkres, Karp/Edmonds; Konig, Es-something

Auction algorithm - Parallel

Cavity method - Statistical mechanics?

\section{Outline of Algorithm}

Assume an $n \times n$ cost matrix.

\begin{enumerate}
	\item Allocate continuous $n^2$ nodes
	\item In $O(n^2)$ time link the nodes together
	\item Sort the $n^2$ nodes in ascending value in $O(n^2 \log n)$ time
	\item Let cursor = minimum valued node
	\item While cursor is not NULL -- This is $O(n)$
	\begin{enumerate}
		\item Assign cursor job to worker
		\item Store the next next cursor
		\item Enqueue (cursor, nil) into Q
		\item While( Q is not empty )  --- This is $O(2n - 1)$
		\begin{enumerate}
			\item Dequeue (q, a) from Q
			\item Set q's left neighbor to point to q's right neighbor (vice versa)
			\item Set q's top neighbor to point to q's bottom neighbor (vice versa)
			\item If a is nil and for each present left, right, top, or bottom neighbor, enqueue (neighbor, direction)
			\item If a is not nill, then if a is present then enqueue (neighbor, a)
			\item Set q's previous to point to q's next (vice versa)
			\item If cursor equals q, update next cursor to equal q's next
			\item delete q
		\end{enumerate}
		\item cursor = next cursor
	\end{enumerate}
\end{enumerate}

Inner loop analysis showing that you end up doing $O(n^2)$ work -- $n$ on the outer loop and $2n - 1$ on the inner loop.

\begin{equation}
	\sum_{k = 1}^{n} 2k - 1 = 2 \frac{n (n + 1)}{2} - n = n^2
\end{equation}

\section{Evaluation}

\section{Conclusion}
\begin{equation}
\begin{pmatrix}
93 & 96 & 87 & \fbox{25} & 103 & 45\\
96 & 102 & 59 & 31 & 72 & \fbox{37}\\
100 & 69 & 73 & 36 & \fbox{50} & 36\\
82 & 46 & \fbox{21} & 78 & 77 & 39\\
92 & \fbox{40} & 72 & 33 & 77 & 45\\
\fbox{39} & 12 & 32 & 68 & 79 & 77
\end{pmatrix}
\end{equation}








\end{document}
